\chapter{Hurwitz existence problem}

\section{Branched coverings of surfaces}

According to standard terminology, a \emph{surface} is simply a topological $2$-manifold. We will, however, only be concerned with compact, connected surfaces without boundary. For the sake of conciseness, unless otherwise stated, we will always implicitly assume that the surfaces we mention have these properties.

There is a structure theorem which completely classifies orientable and non-orientable surfaces.
\begin{itemize}
\item An orientable surface is (homeomorphic to) a connected sum of $g\ge 0$ tori. We call such a connected sum a \emph{surface of genus $g$}, and we denote it by $\surf{g}$. By definition, we say that $\surf{0}$ is the $2$-sphere $\sphere{}$; this is consistent with the formula $\chi(\surf{g})=2-2g$ for the Euler characteristic.
\item A non-orientable surface is (homeomorphic to) a connected sum of $g\ge 1$ real projective planes. We denote such a connected sum by $\nosurf{g}$. The Euler characteristic for a non-orientable surfaces is given by $\chi(\nosurf{g})=2-g$.
\end{itemize}

Loosely speaking, given two surfaces $\Sigma,\tSigma$, a \emph{covering map} between them is a continuous function $\map{f}{\tSigma}{\Sigma}$ which is locally modeled on the identity function $\umap{\RR^2}{\RR^2}$. Sometimes, however, the notion of covering map can be too restrictive. Consider, for instance, the sphere $\sphere{}$: being simply connected, it does not admit any non-trivial coverings. However, every (non-constant) homolorphic function $\umap{\sphere{}}{\sphere{}}$ is \emph{almost} a covering map, in the sense that it is locally modeled on the identity $\umap{\CC}{\CC}$, except for a finite number of \emph{branching points}, where it looks like the map
\[
\Map{F_k}{\CC}{\CC}{z}{z^k}
\]
for some $k\ge 2$. In fact, it turns out that every (non-constant) holomorphic function between two complex $1$-manifolds (which, topologically, are just oriented surfaces), has this remarkable property. This motivates the following definition.

\begin{definition}
Let $\Sigma,\tSigma$ be two surfaces. A continuous function $\map{f}{\tSigma}{\Sigma}$ is a \emph{branched covering map} (or simply a \emph{branched covering}) if the following property holds: for every $x\in\Sigma$, $\wtilde{x}\in f^{-1}(x)$ there exist a positive integer $k$, open neighborhoods $U,\wtilde{U}$ of $x,\wtilde{x}$ respectively, and homeomorphisms $\map{\phi}{U}{\CC}$, $\map{\wtilde{\phi}}{\wtilde{U}}{\CC}$ such that $\phi(x)=0$, $\wtilde{\phi}(\wtilde{x})=0$, $f(\wtilde{U})=U$ and the diagram
\begin{diagram}
\wtilde{U}\rar{f}\dar{\wtilde{\phi}}&U\dar{\phi}\\
\CC\rar{F_k}&\CC
\end{diagram}
commutes ($F_k$ is the map defined above).
\end{definition}

More informally, we can say that a branched covering is a continuous function between surfaces which is locally modeled on the complex map $z\mapsto z^k$, where $k\ge 1$ depends on the point. Note that, for each point $\wtilde{x}\in\tSigma$, the integer $k$ is well-defined, independently of the charts $\phi$ and $\wtilde{\phi}$: using the notation from the definition, we have that $k$ is equal to the cardinality of $f^{-1}(y)\cap\wtilde{U}$, where $y$ is any point in $U\setminus\{x\}$. We call this integer the \emph{branching degree} of $\wtilde{x}$; to emphasize the dependence on $\wtilde{x}$, we will sometimes denote it by $k(\wtilde{x})$.

A point $\wtilde{x}\in\tSigma$ is called a \emph{branching point} if $k(\wtilde{x})>1$; in other words, if $f$ is \emph{not} a local homeomorphism in a neighborhood of $\wtilde{x}$. We also say that a point $x\in\Sigma$ is a \emph{branching point} if $f^{-1}(x)$ contains at least one branching point; usually, no ambiguity will arise as to which kind of branching point we are referring to.

It is not hard to see that branching points are quite rare. If $\wtilde{x}\in\tSigma$ is a branching point, then (using again the notation from the definition) every other point in $\wtilde{U}$ is not a branching point; by compactness, it follows that the set of branching points in $\tSigma$ is finite. As a consequence, the set of branching points in $\Sigma$ is finite as well.

Given a branched covering $\map{f}{\tSigma}{\Sigma}$, we denote by $\holed{\Sigma}$ the subspace of $\Sigma$ containing all the non-branching points; since the set of branching points is finite, $\holed{\Sigma}$ is a non-compact connected surface with finitely many punctures. We also set $\holed{\tSigma}=f^{-1}(\holed{\Sigma})$, and we denote by $\holed{f}$ the restriction of $f$ to $\holed{\tSigma}$. By construction, $\map{\holed{f}}{\holed{\tSigma}}{\holed{\Sigma}}$ is a covering map and, as such, has a well-defined degree $d$ (the number of preimages of an arbitrary point); we call this integer the \emph{degree} of the branched covering $f$. The following proposition shows that the notion of degree extends nicely to branching points.

\begin{proposition}\label{hurwitz:th:sum-of-preimage-degrees}
Let $\map{f}{\tSigma}{\Sigma}$ be a branched covering of degree $d$. Then for every point $x\in\Sigma$ we have that the set $f^{-1}(x)$ is finite and
\[
\sum_{\wtilde{x}\in f^{-1}(x)}k(\wtilde{x})=d.
\]
\end{proposition}
\begin{proof}
If $x$ is not a branching point, the conclusion follows immediately, since $x$ has exactly $d$ preimages, all of which have branching degree $k$ equal to $1$. Assume now that $x$ is a branching point; it is clear from the definition that the set $f^{-1}(x)\subs\tSigma$ is discrete and, hence, finite. Let $f^{-1}(x)=\{\wtilde{x}_1,\ldots,\wtilde{x}_r\}$. Fix open neighborhoods $\wtilde{U}_1,\ldots,\wtilde{U}_r$ of $\wtilde{x}_1,\ldots,\wtilde{x}_r$ respectively as in the definition; a routine compactness argument shows that there exists an open neighborhood $U$ of $x$ such that $f^{-1}(U)\subs \wtilde{U}_1\cup\ldots\cup\wtilde{U}_r$. Fix a point $y\in U\setminus\{x\}$: it follows from the discussion above that $y$ is not a branching point and that $\card{f^{-1}(y)\cup\wtilde{U}_i}=k(\wtilde{x}_i)$ for every $1\le i\le r$. Since $\card{f^{-1}(y)}=d$, we immediately conclude that
\[
\sum_{i=1}^r k(\wtilde{x}_i)=\sum_{i=1}^r\card{f^{-1}(y)\cap\wtilde{U}_i}=\card{f^{-1}(y)}=d.\qedhere
\]
\end{proof}

\section{Branching data}

Let $\map{f}{\tSigma}{\Sigma}$ be a branched covering. For each point $x\in\Sigma$, as we have just seen in \cref{hurwitz:th:sum-of-preimage-degrees}, the sum $k(\wtilde{x}_1)+\ldots k(\wtilde{x}_r)$ of the branching degrees of its preimages is equal to the degree $d$ of the branched covering. Since there is no natural ordering on the set $f^{-1}(x)$, the appropriate combinatorial object for representing the collection $k(\wtilde{x}_1),\ldots,k(\wtilde{x}_r)$ is a \emph{partition}.

\begin{definition}
Let $d$ be a positive integer. A \emph{partition} of $d$ is an unordered finite multiset $\pi=[k_1,\ldots,k_r]$, where $k_i>0$ is an integer for every $1\le i\le r$ and $k_1+\ldots+k_r=d$.
\end{definition}

Given a positive integer $d$, we denote the set of all partitions of $d$ by $\Partitions{d}$. If $\pi=[k_1,\ldots,k_r]$ is a partition of $d$, we call the integer $r$ the \emph{length} (or \emph{size}, or \emph{cardinality}) of $\pi$, and we denote it by $\len{\pi}$. We also introduce a new quantity, the \emph{branching number} $v(\pi)=d-\len{\pi}$, whose purpose will soon become apparent.

For every point $x\in\Sigma$, if $f^{-1}(x)=\{\wtilde{x}_1,\ldots,\wtilde{x}_r\}$, we can define the \emph{associated partition} $\pi(x)=[k(\wtilde{x}_1),\ldots,k(\wtilde{x}_r)]\in\Partitions{d}$. For all non-branching points, the associated partition will simply be $[1,\ldots,1]$. On the contrary, if $x$ is a branching point, then $\pi(x)\neq[1,\ldots,1]$ (or, equivalently, $v(\pi(x))>0$).

\begin{definition}
Let $\map{f}{\tSigma}{\Sigma}$ be a branched covering of degree $d$. Let $x_1,\ldots,x_n\in\Sigma$ be the branching points of $f$. The \emph{branching datum} of $f$ is the tuple
\[
\DD(f)=\datum{\tSigma,\Sigma}{d}{\pi(x_1),\ldots,\pi(x_n)}.
\]
\end{definition}

Branching data are a way to extract some combinatorial information from branched coverings. Even though the exact location of the branching points (both in $\Sigma$ and in $\tSigma$) is not encoded in the branching datum, this piece of information is completely irrelevant, since surfaces are \emph{homogeneous}\footnote{By \emph{homogeneous}, we mean that the group of homeomorphisms of a surface $\Sigma$ acts transitively on the set of $n$-uples of points $(y_1,\ldots,y_n)$.}. However, the reader should not be induced to believe that the combinatorial information provided by the branching datum is enough to fully reconstruct the topology of $f$. In fact, it turns out that this is not the case\todo{More on this later (?).}.

It is now perfectly legitimate to wonder 