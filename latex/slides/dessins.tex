\section{\texorpdfstring{\Dessins{}}{Dessins d'enfant}}

\subsection{Costruzione}
\begin{frame}[fragile]
\begin{columns}[onlytextwidth]
\column{.65\textwidth}
Sia $\map{f}{\tSigma}{\sphere}$ un rivestimento ramificato con punti di ramificazione $x,y,z\in\sphere$.
\begin{itemize}
\def\ee{\textcolor{green!60!black}{e}}
\item Tracciamo un arco $\ee$ fra $x$ e $y$.
\item $\Gamma= f^{-1}(\ee)$ è un grafo su $\tSigma$.
\item Se coloriamo $f^{-1}(x)$ di nero e $f^{-1}(y)$ di bianco, $\Gamma$ è bipartito.
\item Il grado di $\wtilde{x_i}\in f^{-1}(x)$ è $k(\wtilde{x}_i)$.
\item Ogni componente \sidenote<1>{connessa}{\draw (mymark) to[out=-110,in=180] (.2,-.9) node[right] {regione complementare};} $\wtilde{D}_i$ di $\tSigma\setminus\Gamma$ contiene esattamente un punto $\wtilde{z}_i\in f^{-1}(z)$.
%\item La restrizione $\map{f}{\wtilde{D}_i}{\sidenotehighlight<1>{\sphere{}\setminus \ee}{\draw (mymark) to[out=90,in=180] (.5,.5) node[right] {disco};}}$\; è un rivestimento ramificato con un solo punto di ramificazione.
\item $\wtilde{D}_i$ è un disco, e ci sono $2k(\wtilde{z}_i)$ \sidenotehighlight<1>{archi}{\draw (mymark) to[out=-90,in=0] (-.6,-.8) node[left] {contati con molteplicità};} lungo il suo perimetro.
\end{itemize}
\column{.35\textwidth}
\begin{center}
\tikzset{
black vertex/.pic={\fill[black] circle(1.5pt);},
white vertex/.pic={\filldraw[line width=.5pt,white,draw=black] circle(1.5pt);},
third vertex/.pic={\draw[scale=1.5,black!70,thick] (45:1pt) -- (-135:1pt) (-45:1pt) -- (135:1pt);}
}
\tikzsetnextfilename{slides-dessin-graph-preimage}
\scalebox{.75}{
\begin{tikzpicture}[x={(1.5,0,0)},y={(0,1.5,0)},z={(.6,.4)},
arc/.style={draw=green,preaction={draw=white,line width=2pt}},
declare function={interp(\a,\b,\t)=\a+\t*(\b-\a);},
pics/draw arc/.style 2 args={code={
\draw[arc] let \p1=(#1),\p2=(#2) in plot[smooth,variable=\t,domain=0:1] ({interp(\x1,\x2,\t)/1.5cm},{interp(\y1,\y2,\t)/1.5cm},{.1*sin(360*\t)});
}},
]
\foreach \i in {1,2,3} {
\coordinate (x-\i) at (0,{1+.5*(\i-1)});
}
\foreach \i in {1,2} {
\coordinate (y-\i) at (1,{1+1/6+2/3*(\i-1)});
}

\pic{draw arc={0,0}{1,0}};
\node[teal!60!black,below] at (.5,0) {$e$};
\foreach \i/\j in {1/1,1/2,2/1,3/1} {\pic{draw arc={x-\i}{y-\j}};}
\draw[violet,-latex,shorten >=.1cm] (x-3) -- (0,0);
\draw[violet,-latex,shorten >=.1cm] (y-2) -- (1,0) node[right,pos=.66] {$f$};
\path (0,0) pic{black vertex} node[below left] {$x$};
\path (1,0) pic{white vertex} node[below right] {$y$};
\foreach \i in {1,2,3} {\path (x-\i) pic {black vertex} node[left] {$\wtilde{x}_{\i}$};}
\foreach \i in {1,2} {\path (y-\i) pic {white vertex} node[right] {$\wtilde{y}_{\i}$};}
\end{tikzpicture}
}\\
\tikz\draw[gray!50] (0,0) -- (2,0);\\[1.5em]
\tikzsetnextfilename{slides-dessin-disk-perimeter}
\tdplotsetmaincoords{65}{-30}
\renewcounter{branchedcoveringcounter}
\scalebox{.75}{
\begin{tikzpicture}[tdplot_main_coords,x={(1.2,0,0)},y={(0,1.2,0)},z={(0,0,1.2)}]
\def\a{30};
\def\yy{{sin(\a)*.7},{-cos(\a)*.7}}

\begin{scope}[shift={(0,0,-2.5)}]
\fill[disk 1] circle (1);
\draw[line width=\edgelinewidth,teal] (-30:1) arc(-30:150:1);
\draw[line width=\edgelinewidth,purple] (150:1) pic{black vertex} node[black,below left] {$x$} arc(150:330:1) pic{white vertex} node[black,below right] {$y$};
\pic{third vertex} node[below left] {$z$};
\end{scope}

\draw[violet,-{Latex[]},shorten >=.1cm] (0,0,0) -- (0,0,-2.5) node[midway,left=2pt] {$f$};

\setcounter{branchedcoveringcounter}{0}
\path[decorate,decoration={show path construction,curveto code={
\filldraw[line width=.1pt,disk 1] (0,0,0) -- (\tikzinputsegmentfirst) -- (\tikzinputsegmentlast) -- cycle;
\tikzmath{
\v=mod(\value{branchedcoveringcounter},30);
\col=ifthenelse(\v>=5&&\v<20,"teal","purple");
}
\draw[\col,line width=\edgelinewidth] (\tikzinputsegmentfirst) .. controls (\tikzinputsegmentsupporta) and (\tikzinputsegmentsupportb) .. (\tikzinputsegmentlast);
\ifnumcomp{\the\value{branchedcoveringcounter}}{=}{0}{
	\draw[black] (0,0,0) -- (\tikzinputsegmentfirst);
}{}
\tikzmath{
if \v==5 then{
{\pic at (\tikzinputsegmentfirst) {white vertex};};
};
if \v==20 then{
{\pic at (\tikzinputsegmentfirst) {black vertex};};
};
}
\stepcounter{branchedcoveringcounter}
}}] plot[smooth,samples={61},variable=\t,domain=-360:360] ({sin(\t)},{-cos(\t)},{\t/360});
\draw[black] (0,0,0) -- (0,-1,1);
\pic{third vertex} node[below left] {$\wtilde{z}_i$};
\end{tikzpicture}}
\end{center}
\end{columns}
\end{frame}

\subsection{Criterio di realizzabilità}
\begin{frame}
Un \emph{\dessin{}} su $\tSigma$ è un grafo bipartito $\Gamma\subs\tSigma$ le cui regioni complementari sono dischi.
\begin{center}
\tikzsetnextfilename{slides-dessin-first-example}
\begin{tikzpicture}[tdplot_main_coords,baseline=0pt,declare function={interp(\a,\b,\t)=\a+\t*(\b-\a);},every plot/.style={smooth,samples=\torusprecision}]
\pgfsetlayers{main,graph vertex}
\draw[surf boundary,fill=disk 1,even odd rule,3d torus];
\draw[surf boundary,3d torus stretch];
\fill[disk 2] [torus graph small region];
\draw[black edge dashed] [torus graph behind];
\drawtorusgraph
\end{tikzpicture}\hspace{2em}
$\sidenote<1>{\DD}{\draw (mymark) to[out=-120,in=0] (-.9,-.5);}=\datum{\surf{1}}{7}{[3,4],[2,2,3],[1,6]}$
\end{center}
\begin{mybox}
Un dato $\DD=\datum{\tSigma}{d}{\pi_1,\pi_2,\pi_3}$ è realizzabile se e solo se esiste un \dessin{} $\Gamma\subs\tSigma$ tale che:
\begin{enumroman}\def\myarrow{\tikz[baseline=-.25em]\draw[violet,line width=1pt,tip/.tip={Latex[sharp,scale=.75]},tip-tip] (0,0) -- (.6cm,0);\;}
\item $\pi_1$ \myarrow gradi dei vertici neri;
\item $\pi_2$ \myarrow gradi dei vertici bianchi;
\item $\pi_3$ \myarrow semiperimetri dei dischi complementari.
\end{enumroman}
\end{mybox}
\end{frame}

\subsection{Un esempio di dato eccezionale}
\begin{frame}
\emph{Esempio}. Il dato $\DD=\datum{\sphere}{4}{[2,2],[2,2],[1,3]}$ è eccezionale. Se non lo fosse, esisterebbe un \dessin{} $\Gamma\subs\sphere$ tale che:
\begin{enumroman}
\item i due vertici neri abbiano gradi $[2,2]$;
\item i due vertici bianchi abbiano gradi $[2,2]$;
\item i due dischi complementari abbiano semiperimetri $[1,3]$.
\end{enumroman}
Tuttavia esiste un unico \dessin{} su $\sphere$ che soddisfa le condizioni \tikzenumlabel{i} e \tikzenumlabel{ii}, e i suoi dischi complementari hanno semiperimetri $[2,2]$.
\begin{center}
\tikzsetnextfilename{slides-dessin-example-sphere}
\begin{tikzpicture}[x={(.8,0)},y={(0,.8)}]
\pgfsetlayers{main,surf edge behind,graph vertex}
\filldraw[disk 1,surf boundary] circle (2);
\fill[scale=1.2,shift={(0,.1)},disk 2,postaction={surf edge={behind}{black edge}}] (45:1) pic{black vertex} to[bend right] (135:1) pic{white vertex} to[bend right] (-135:1) pic{black vertex} to[bend right] (-45:1) pic{white vertex} to[bend right] cycle;
\draw[surf boundary dashed,opacity=.4] (2,0) arc (0:-180:2 and {2*cos(70)});
\end{tikzpicture}
\end{center}
\end{frame}